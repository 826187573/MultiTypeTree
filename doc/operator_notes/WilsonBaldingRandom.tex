\documentclass[a4paper,11pt]{article}

\usepackage{tikz}
\usetikzlibrary{calc}

\usepackage{amsmath}

\title{Coloured Wilson-Balding Random\\%
acceptance probability calculations}

\begin{document}

\maketitle

\section{Non-root move}

\begin{figure}
\begin{center}
\begin{tikzpicture}
\coordinate (leaf1) at (-4,0);
\coordinate (leaf2) at (-1,0);
\coordinate (leaf3) at (1,0);
\coordinate (leaf4) at (4,0);
\coordinate (internal) at (0,1);
\coordinate (root) at (0,4);

\draw (leaf2)--(internal) node[below]{$i$}--(leaf3);
\draw (leaf1) node[right=3pt]{$i_s$}--(root) node[left=3pt]{$i_g$} node[right=3pt]{$j_p$}--(leaf4) node[right]{$j$};
\fill (root) circle (0.05);
\fill (internal) circle (0.05);
\fill (leaf1) circle (0.05);
\fill (leaf2) circle (0.05);
\fill (leaf3) circle (0.05);
\fill (leaf4) circle (0.05);

% AUCTeX preview can't handle this:
% \coordinate (source) at ($(leaf1)!0.6!(root)$);
% \coordinate (dest) at ($(leaf4)!0.5!(root)$);
% \draw (internal)--(source);
% \draw [dashed] (internal)--(dest);
% \fill (source) circle (0.05);
% \fill (dest) circle (0.05);

\end{tikzpicture}
\end{center}
\caption{Non-root move}
\end{figure}

The non-root move acts on a state vector $x=\{T,n,t,c\}$ where $T$ is
the age of $i_p$, $n$ is the number of migration events along the
length of the branch, $t$ is the vector containing the times of those
events relative to the age of $i$ and $c$ is a vector containing the
destination colours of these events.

Note that here we identify a particular move by the nodes $i$ and $j$
chosen to be the base of the source and destination branches
respectively (i.e. the topology change to be implemented). As these
nodes are always chosen uniformly at random from the nodes of the
tree, the probabilities of their selection do not factor into the
acceptance probability calculation.

In the non-root case, the CWBR move draws a new state
$x'=\{T',n',t',c'\}$ from the following proposal density:
\begin{align}
  q(x'|x) &= P(T',n',t',c'|x)\nonumber\\
&= P(T')P(n'|T')P(t'|n')P(c|n')
\end{align}

\subsection{Proposal for $i'_p$ age}

The age $T'$ of the new attachment point $i'_p$ is selected by drawing directly from
$U(T_{\min},T_{\max})$, where $T_{\min}=\max(T_i,T_{j_p})$ and
$T_{\max}=T_{j_p}$ are defined in terms of the ages $T_i$ and
$T_{j_p}$ of nodes $i$ and $j_p$ respectively. (As these node heights
are not modified by the operator, we do not consider them part of the
state here.) Thus,
\begin{equation}
  P_{\bar{T}}(T') = \left\{\begin{array}{rl}
      \frac{1}{T_{\max}-T_{\min}} & \text{ for } T'\in[T_{\min},T_{\max}]\\
      0 & \text{ otherwise}\end{array}\right.
\end{equation}

\subsection{Migration count proposal}

The new migration count $n'$ is chosen from a Poissonian distribution
with a mean of $\mu (T'-T_i)$, where $\mu$ is a tuning parameter:
\begin{equation}
  P(n'|T') = e^{-\mu L'}\frac{(\mu L')^{n'}}{n'!}
\end{equation}
where we have used $L'\equiv T'-T_i$.

\subsection{Migration times proposal}

The new migration times are chosen by drawing $n'$ values $\tau'$
independently from $U(T_i,T')$, then sorting them from smallest to
largest. Defining the sorting function $t'=S(\tau')$ we write
\begin{equation}
  P(t'|n') = \int_{[T_i,T']^{n'}}d^{n'}\tau'\delta^{n'}(t'-S(\tau'))(L')^{-n'}.
\end{equation}
We can rewrite the delta function term as
\begin{equation}
  \delta^{n'}(t'-S(\tau'))=\sum_{\alpha=1}^{n'!}\delta^{n'}(t'_{\alpha}-\tau')
\end{equation}
where $t'_{\alpha}$ are distinct permutations of $t'$. Therefore
\begin{align}
  P(t'|n')&=\sum_{\alpha=1}^{n'!}\int_{[T_i,T']^{n'}}d^{n'}\tau'\delta^{n'}(t'_{\alpha}-\tau')(L')^{-n'}\nonumber\\
&=\frac{n'!}{(L')^{n'}}
\end{align}

\subsection{Migration colours proposal}

Every migration event results in a change of colour at that point on
the branch.  Each new colour is selected at random from those
remaining after the present colour is excluded. That is,
\begin{equation}
  P(c'|n') = \frac{1}{(N-1)^{n'}}
\end{equation}
where $N$ is the total number of available colours.

\subsection{Full proposal density}

The expression for the full proposal is then:
\begin{align}
  q(x'|x) &=\frac{1}{T_{\max}-T_{\min}}I(T'\in[T_{\min},T_{\max}])
  \times e^{-\mu L'}\frac{(\mu L')^{n'}}{n'!}
  \times \frac{n'!}{(L')^{n'}}
  \times \frac{1}{(N-1)^{n'}}\nonumber\\
&=\left\{\begin{array}{rl}
    \frac{\exp[-\mu(T'-T_i)]}{T_{\max}-T_{\min}}\left(\frac{\mu}{N-1}\right)^{n'} &
    \text{ for }T'\in[T_{\min},T_{\max}]\\
    0 & \text{ otherwise}
    \end{array}\right.
\label{eq:nonRootDensity}
\end{align}
where again $T_{\min}=\max(T_i,T_j)$ and $T_{\max}=T_{j_p}$.

\subsection{Acceptance probability}

For the purpose of determining the proposal density of the reverse
move we must use $j\rightarrow i_s$ and $j_p\rightarrow i_g$ in
eq.~(\ref{eq:nonRootDensity}). The acceptance probability is then
\begin{align}
  \alpha(x'|x) &=
  \min\left[1,\frac{q(x|x')\pi(x')}{q(x'|x)\pi(x)}\right]\nonumber\\
&=\min\left[1,\frac{T_{j_p}-\max(T_i,T_j)}{T_{i_g}-\max(T_i,T_{i_s})}e^{-\mu(T-T')}\left(\frac{\mu}{N-1}\right)^{n-n'}\frac{\pi(x')}{\pi(x)}\right].
\end{align}
provided $q(x'|x)$ is non-zero, as it must be if $x'$ has been sampled
from $q(x'|x)$.  The condition on $T$ ensures the move will never be
accepted unless the reverse move is possible.

\section{Forward root move}

\begin{figure}
\begin{center}
\begin{tikzpicture}
\coordinate (leaf1) at (-2,0);
\coordinate (leaf2) at (0,0);
\coordinate (leaf3) at (4,0);
\coordinate (oldRoot) at (2,3);
\coordinate (newRoot) at (0,5);

\draw (leaf2) node[left]{$i_s$}--(oldRoot) node[right]{$j$}--(leaf3);
\draw [dashed] (leaf1) node[left]{$i$}--(newRoot)--(oldRoot);

% AUCTeX preview can't handle this:
% \coordinate (source) at ($(leaf2)!0.6!(oldRoot)$);
% \draw [color=gray](leaf1)--(source);
% \fill (source) circle (0.05);

\fill (oldRoot) circle (0.05);
\fill (newRoot) circle (0.05);
\fill (leaf1) circle (0.05);
\fill (leaf2) circle (0.05);
\fill (leaf3) circle (0.05);

\end{tikzpicture}
\end{center}
\caption{The forward root move, with the dashed lines representing
the new branches and the grey line representing the branch to be removed.}
\end{figure}

There are two distinct moves involving the root. The ``forward'' move,
in which node $j$ is the root of the tree in the current state, and
the ``backward'' move: in which $i$ is a daughter of the root in the
current state.  We consider firstly the forward move.

For the forward move, the relevant components of the starting state
are contained in $x=\{T,n,t,c\}$, where these variables are defined as
in the previous section.  Unlike in the case of the non-root move, the
proposal state has the distinct form
$x'=\{T';n',t',c';n'_s,t'_s,c'_s\}$, where the additional variables
$n'_s$, $t'_s$ and $c'_s$  represent the number, times and colours
of migration events along the new sister branch of $\langle
i,i_p'\rangle$: $\langle j,i_p'\rangle$.

The proposal density has the following form:
\begin{align}
  q(x'|x)&=P(T',n',t',c',n'_s,t'_s,c'_s|x)\\
&=P(T')P(n',n'_s|T')P(t'|n',T')P(t'_s|n'_s,T')P(c',c'_s|n',n'_s)\nonumber
\end{align}

\subsection{Root height proposal}

The height of the proposed new root, $T'$ is selected by drawing
$a\sim g$ where
\begin{equation}
g(a)=\frac{\alpha}{T_j}e^{-\frac{\alpha}{T_j}a}
\end{equation}
then setting $T'=T_j+a$. Therefore
\begin{align}
  P(T')&=\int_0^\infty da\, \delta(T'-(T_j+a))g(a)\nonumber\\
&=\frac{\alpha}{T_j}\exp[-\frac{\alpha}{T_j}(T'-T_j)]
\end{align}

\subsection{Migration count proposal}

The numbers of new migration events along the new branches above $i$
and its new sister are chosen from Poissonian distributions with means
$\mu(T'-T_i)$ and $\mu(T'-T_j)$.  Thus,
\begin{equation}
  P(n',n'_s)=e^{-\mu L'}\frac{(\mu L')^{n'}}{n'!}
  \times e^{-\mu L'_s}\frac{(\mu L'_s)^{n'_s}}{n'_s!}
\end{equation}
where we have defined $L'\equiv T'-T_i$ and $L'_s\equiv T'-T_j$.

\subsection{Migration times proposal}

The times of the new migration events along the new branch above $i$
are chosen by independently selecting $n'$ times from $U(T_i,T')$ then
sorting them from smallest to largest.  The unsorted times are
contained in the vector $\tau'$ and we again use the sorting function
$S$ to write $t'=S(\tau')$.  We therefore have
\begin{equation}
  P(\tau'|n',T')=\int_{[T_i,T']^{n'}}d^{n'}\tau\delta^{n'}(t'-S(\tau'))(L')^{-n'}.
\end{equation}
Again expanding the delta function in terms of permutations of $t'$,
we find
\begin{align}
P(\tau'|n',T')&=\sum_{\alpha=1}^{n'!}\int_{[T_i,T']^{n'}}d^{n'}\tau\delta^{n'}(t'_\alpha-\tau')(L')^{-n'}\nonumber\\
&=\frac{n'!}{(L')^{n'}}.
\end{align}
Similarly, we select $n'_s$ times from $U(T_j,T')$ and set
$t'_s=S(\tau'_s)$, yielding
\begin{equation}
  P(\tau'_s|n'_s,T')=\frac{n'_s!}{(L'_s)^{n'_s}}.
\end{equation}

\subsection{Migration colours proposal}

Each new migration results in the lineage shifting to a new colour,
randomly chosen from a uniform distribution over the colours available
after the previous colour of the lineage has been excluded.  Thus,
\begin{equation}
  P(c',c'_s|n',n'_s)=\frac{1}{(N-1)^{n'+n'_s}}.
\end{equation}

\subsection{Full proposal density}

Combining the conditional distributions and densities above yields
\begin{align}
  q(x'|x)=&\frac{\alpha}{T_j}\exp\left[-\frac{\alpha}{T_j}(T'-T_j)\right]
  \times e^{-\mu L'}\frac{(\mu L')^{n'}}{n'!}
  \times e^{-\mu L'_s}\frac{(\mu L'_s)^{n'_s}}{n'_s!}\nonumber\\
  &\times \frac{n'!}{(L')^{n'}}
  \times \frac{n'_s!}{(L'_s)^{n'_s}}
  \times (N-1)^{-(n'+n'_s)}\nonumber\\
=&\frac{\alpha}{T_j}\exp\left[-\mu(2T'-T_i-T_j)-\alpha\left(\frac{T'}{T_j}-1\right)\right]\left(\frac{\mu}{N-1}\right)^{n'+n'_s},
\end{align}
or $0$ if $T'<T_j$.

\section{Backward root move}

\begin{figure}
\begin{center}
\begin{tikzpicture}
\coordinate (leaf1) at (-2,0);
\coordinate (leaf2) at (-0,0);
\coordinate (leaf3) at (4,0);
\coordinate (newRoot) at (2,3);
\coordinate (oldRoot) at (0,5);

\draw (leaf2) node[left]{$j$}--(newRoot)--(leaf3);
\draw [color=gray] (leaf1)
node[left,color=black]{$i$}--(oldRoot)--(newRoot) node[right,color=black]{$i_s$};
\fill (newRoot) circle (0.05);
\fill (oldRoot) circle (0.05);
\fill (leaf1) circle (0.05);
\fill (leaf2) circle (0.05);
\fill (leaf3) circle (0.05);

% AUCTeX preview can't handle this:
% \coordinate (source) at ($(leaf2)!0.6!(newRoot)$);
% \draw [dashed] (leaf1)--(source);
% \fill (source) circle (0.05);

\end{tikzpicture}
\end{center}
\caption{The backward root move, with the dashed lines representing
the new branches and the grey line representing the branch to be
removed.}
\end{figure}

The starting and proposal states for the backward root move are simply
the reverse of those in the forward root move. That is,
$x=\{T;n,t,c;n_s,t_s,c_s\}$ and $x'=\{T';n',t',c'\}$. The proposal
density has the following form:
\begin{align}
  q(x'|x) &= P(T',n',t',c'|x)\nonumber\\
&= P(T')P(n'|T')P(t'|n',T')P(c'|n')
\end{align}

\subsection{Proposal for $i'_p$ age}

The age $T'$ of the new node $i_p'$ is drawn from
$U(T_{\min},T_{\max})$, where $T_{\min}=\max(T_i,T_j)$ and $T_{\max}=T_{j_p}$. Thus,
\begin{equation}
  P(T')=\left\{\begin{array}{rl}
\frac{1}{T_{\max}-T_{\min}} & \text{ for }T'\in[T_{\min},T_{\max}]\\
0 & \text{ otherwise}\end{array}\right.
\end{equation}

\subsection{Proposal for migration count}

The migration count along the new branch is drawn from
\begin{equation}
  P(n'|T')=e^{-\mu(T'-T_i)}\frac{(\mu(T'-T_i))^{n'}}{n'!}.
\end{equation}

\subsection{Proposal for migration times}

The migration times are again distributed according to
\begin{equation}
  P(t'|n',T')=\frac{n'!}{(T'-T_i)^{n'}}.
\end{equation}

\subsection{Proposal for migration colours}
The migration colours are distributed according to
\begin{equation}
  P(c'|n')=\frac{1}{(N-1)^{n'}}.
\end{equation}

\subsection{Full proposal density}

The combined proposal density for the backwards move is
\begin{align}
  q(x'|x) &= \frac{1}{T_{j_p}-T_{\min}}I(T'\in[T_{\min},T_{j_p}])
\times e^{-\mu(T'-T_i)}\frac{(\mu(T'-T_i))^{n'}}{n'!}
\times \frac{n'!}{(T'-T_i)^{n'}}
\times (N-1)^{-n'}\nonumber\\
&=\frac{\exp[-\mu(T'-T_i)]}{T_{\max}-T_{\min}}\left(\frac{\mu}{N-1}\right)^{n'}
I(T'\in[T_{\min},T_{\max})])
\end{align}
where again $T_{\min}=\max(T_i,T_j)$ and $T_{\max}=T_{j_p}$.

\section{Acceptance probability for root moves}

The assymetry of the forward and backward root causes their acceptance
probabilities to also differ.

\subsection{Forward root move}

For the forward move, the current state is $x=\{T,n,t,c\}$ and the
proposal state $x'=\{T',n',t',c',n'_s,t'_s,c'_s\}$. The reverse move
is the backward root move. Note that for the reverse move,
$j\rightarrow i_s$ and $j_p\rightarrow i_g$ where $i_s$ and $i_g$ are
the sister and grandparent nodes of $i$. Since we will always have
$T\in[\max(T_i,T_{i_s}),T_{i_g}]$, the Hastings ratio is
\begin{align}
  \frac{q(x|x')}{q(x'|x)}
=& \frac{\exp[-\mu(T-T_i)]}{T_{i_g}-\max(T_i,T_{i_s})}\left(\frac{\mu}{N-1}\right)^n\\
&\times
\frac{T_j}{\alpha}\exp\left[\mu(2T'-T_i-T_j)+\alpha\left(\frac{T'}{T_j}-1\right)\right]\left(\frac{\mu}{N-1}\right)^{-(n'+n'_s)}\nonumber\\
=& \frac{T_j\exp\left[-\mu(T+T_j-2T')+\alpha\left(\frac{T'}{T_j}-1\right)\right]}{\alpha(T_{i_g}-\max(T_i,T_{i_s}))}\left(\frac{\mu}{N-1}\right)^{n-(n'+n'_s)}.\nonumber
\end{align}

\subsection{Backward root move}

For the backward move, the current state is $x=\{T,n,t,c,n',t',c'\}$
and the proposal state is $x'=\{T',n',t',c'\}$. The reverse move is
the forward root move. For the reverse move $j\rightarrow i_s$. Since
by definition we will have always have $T>T_{i_s}$, the Hastings ratio
will be
\begin{align}
  \frac{q(x|x')}{q(x'|x)}
=&
\frac{\alpha}{T_{i_s}}\exp\left[-\mu(2T-T_i-T_{i_s})-\alpha\left(\frac{T}{T_{i_s}}-1\right)\right]\left(\frac{\mu}{N-1}\right)^{n+n_{i_s}}\nonumber\\
&\times
\frac{T_{j_p}-\max(T_i,T_j)}{\exp[-\mu(T'-T_i)]}\left(\frac{\mu}{N-1}\right)^{-n'}\nonumber\\
&=\frac{\alpha(T_{j_p}-\max(T_i,T_j))}{T_{i_s}}\left(\frac{\mu}{N-1}\right)^{(n+n_{i_s})-n'}\nonumber\\
&\times\exp\left[-\mu(2T-T_{i_s}-T')-\alpha\left(\frac{T}{T_{i_s}}-1\right)\right].
\end{align}


\end{document}
