\documentclass[a4paper,11pt]{article}

\usepackage{tikz}
\usetikzlibrary{calc}

\usepackage{amsmath}

\title{Coloured Wilson-Balding Random\\%
acceptance probability calculations}

\begin{document}

\maketitle

\section{Non-root move}

\begin{figure}
\begin{center}
\begin{tikzpicture}
\coordinate (leaf1) at (-4,0);
\coordinate (leaf2) at (-1,0);
\coordinate (leaf3) at (1,0);
\coordinate (leaf4) at (4,0);
\coordinate (internal) at (0,1);
\coordinate (root) at (0,4);

\draw (leaf2)--(internal) node[below]{$i$}--(leaf3);
\draw (leaf1)--(root) node[right=3pt]{$j_p$}--(leaf4) node[right]{$j$};
\fill (root) circle (0.05);
\fill (internal) circle (0.05);
\fill (leaf1) circle (0.05);
\fill (leaf2) circle (0.05);
\fill (leaf3) circle (0.05);
\fill (leaf4) circle (0.05);

% AUCTeX preview can't handle this:
%\coordinate (source) at ($(leaf1)!0.6!(root)$);
%\coordinate (dest) at ($(leaf4)!0.5!(root)$);
%\draw (internal)--(source);
%\draw [dashed] (internal)--(dest);
%\fill (source) circle (0.05);
%\fill (dest) circle (0.05);

\end{tikzpicture}
\end{center}
\caption{Non-root move}
\end{figure}

The non-root move acts on a state vector $x=\{T,n,t,c\}$ where $T$ is
the length of the branch between the chosen node $i$ and its parent,
$n$ is the number of migration events along the length of the branch,
$t$ is the vector containing the times of those events relative to the
age of $i$ and $c$ is a vector containing the destination colours of
these events.

Note that here we identify a particular move by the nodes $i$ and $j$
chosen to be the base of the source and destination branches
respectively. As these nodes are always chosen uniformly at random from the
nodes of the tree, the probabilities do not factor into the acceptance
probability calculation.

In the non-root case, the CWBR move draws a new state
$x'=\{T',n',t',c'\}$ from the following proposal density:
\begin{align}
  q(x'|x,\bar{T}) &= P(T',n',t',c'|x)\nonumber\\
&= P_{\bar{T}}(T')P(n'|T')P(t'|n')P(c|n')
\end{align}
Here $\bar{T}$ represents the age of the parent $j_p$ relative to the
age of the chosen source node $i$.

\subsection{Branch length proposal}

The new branch length $T'$ is selected by drawing directly from
$U(0,\bar{T})$. Thus,
\begin{equation}
  P_{\bar{T}}(T') = \left\{\begin{array}{rl}
      \frac{1}{\bar{T} & \text{ for } T'\in[0,\bar{T}]\\
      0 & \text{ otherwise}\end{array}\right.}
\end{equation}

\subsection{Migration count proposal}

The new migration count $n'$ is chosen from a Poissonian distribution
with a mean of $\mu T'$, where $\mu$ is a tuning parameter:
\begin{equation}
  P(n'|T') = e^{-\mu T'}\frac{(\mu T')^{n'}}{n'!}
\end{equation}

\subsection{Migration times proposal}

The new migration times are chosen by drawing $n'$ values $\tau'$
independently from $U(0,T')$, then sorting them from smallest to
largest. Defining the sorting function $t'=S(\tau')$ we write
\begin{equation}
  P(t'|n') = \int_{[0,T']^{n'}}d^{n'}\tau'\delta^{n'}(t'-S(\tau'))(T')^{-n'}.
\end{equation}
We can rewrite the delta function term as
\begin{equation}
  \delta^{n'}(t'-S(\tau'))=\sum_{\alpha=1}^{n'!}\delta^{n'}(t'_{\alpha}-\tau')
\end{equation}
where $t'_{\alpha}$ are distinct permutations of $t'$. Therefore
\begin{align}
  P(t'|n')&=\sum_{\alpha=1}^{n'!}\int_{[0,T']^{n'}}d^{n'}\tau'\delta^{n'}(t'_{\alpha}-\tau')(T')^{-n'}\nonumber\\
&=\frac{n'!}{(T')^{n'}}
\end{align}

\subsection{Migration colours proposal}

Every migration event results in a change of colour at that point on
the branch.  Each new colour is selected at random from those
remaining after the present colour is excluded. That is,
\begin{equation}
  P(c'|n') = \frac{1}{(N-1)^{n'}}
\end{equation}
where $N$ is the total number of available colours.

\subsection{Full proposal}

The expression for the full proposal is then:
\begin{align}
  q(x'|x) &=\frac{1}{\bar{T}}I(T'\in[0,\bar{T}])
  \times e^{-\mu T'}\frac{(\mu T')^{n'}}{n'!}
  \times \frac{n'!}{(T')^{n'}}
  \times \frac{1}{(N-1)^{n'}}\nonumber\\
&=\left\{\begin{array}{rl}
    \frac{1}{\bar{T}}e^{-\mu T'}\left(\frac{\mu}{N-1}\right)^{n'} &
    \text{ for }T'\in[0,\bar{T}]\\
    0 & \text{ otherwise}
    \end{array}\right.
\end{align}

\section{Root move}


\end{document}
